
\chapter{绪论}
\label{chap:introduction}

\section{研究的背景与意义}
	90年代互联网技术飞速发展时期,随着Web使用率的快速增长,传统的WEB架构曾暴露出可扩展性差、效率低下等严重局限。而REST架构的提出,为WEB提供了一种全新的完整的架构约束,REST强调组件交互的可扩展性、接口的通用性和组件间的解耦性,解决了传统WEB架构的上述弊端。

本课题研究了REST架构的要解决的问题、使用的方法和效果,从REST的缓存、统一接口、系统分层等几个约束出发,逐步设计和实现一个基于REST架构、包含功能和测试组件的系统案例,并从REST系统的可扩展性、解耦性等性质进行定量和定性的分析。REST架构本身具有很强的理论性和抽象性,而本课题通过对REST抽象架构的实例化,提出了一种分析REST架构的方法和见解。